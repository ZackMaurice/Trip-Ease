\documentclass[12pt]{article}

\usepackage{graphicx}
\usepackage{subcaption}

\title{Bencozaham}
\author{
Benjamin Karstad
\texttt{- 101092172}
\and
Conor MacQuarrie
\texttt{- 100824788}
\and
Zacharaia Maurice
\texttt{- 101116412}
\and
Graham Shannon
\texttt{- 100970902}
}

\begin{document}

	\maketitle

	\section*{Project Description}

	Our project is a companion app that will help keep you organized when travelling.
	Our app is interesting because it eases some of the stressful components of travelling.
	It makes planning for packing easy.
	It takes away the stress of not knowing whether you forgot something at your destination.
	It lets you spend responsibly when you may not have a great perception of the value of local currency.
	It helps you organize your daily events during your trip.

	While all of the features we are implementing can already be found,
	having everything in one place means that the minutia of traveling are streamlined and consolidated.
	All of the core features will also be able to be saved locally, meaning that even when you do not have an internet connection,
	you can be sure that you have all the necessary tools for a successful vacation or business trip.

	The project makes sense in a mobile form because most people tend to bring their mobile devices with them on vacation.
	Alerting users with important information is also better on mobile devices;
	since people tend to have their phone on them most of the day so they will see the alert promptly.

	\pagebreak

	\section*{Functional Properties}

	\begin{itemize}

		\item{Packing Checklist}

		The app will have a checklist of items that you packed.
		This makes it easier to know what you have and means that you won't forget to pack anything when you leave.
		You will also be able to load from common templates.
		If you're packing for a camping trip, the app will suggest items like a tent, sleeping bag, cooler etc.
		If you're packing for a beach vacation the app will suggest items like a bathing suit, sunglasses and sunscreen.
		You will be able to save your own templates to make packing for similar trips in the future easy.

		\item{Itinerary/Planner}

		This will allow you to schedule flights and any important events.
		You can choose to set reminders for these events, and the reminders can include any packed items you wish to bring.
		For example, your flight reminder may also remind you to have your passport.
		A reminder for a hike may remind you to bring your camera, hiking boots, and your water bottle.

		\item{Budgeting}

		This will allow you to give your entire trip a budget.
		This budget will be converted to the destination currency.
		This will give you an idea of how much money you can spend on your trip,
		while staying within your intended budget.
		You can also input a specific amount of currency.

		The user will also have access to a database of common goods/services and their approximate price in the destination country.
		This ensures that travellers know what they should be paying and can avoid being gouged by tourist traps.

		\item{Misc Utilities}

		We also plan to add functionality for an array of miscellaneous convenience tools,
		little things that a traveller might need, all located in one app.
		Some things that may be implemented are:

		\begin{itemize}
			\item Weather
			\item Local emergency numbers
			\item Translation \& Common Phrases
		\end{itemize}

		\item{Offline Saving}

		All of the core functionality can optionally be stored locally on a user's device (where feasible).
		This is important because reliable internet access is not always guaranteed in a travelling environment,
		cloud storage is a great tool, but knowing that a shoddy cell connection will not revoke access to all your
		travel info affords peace of mind and security.

	\end{itemize}

	\section*{Non-Functional Properties}


	\begin{enumerate}

		\item{User-friendly and accessible user interface}
		\begin{itemize}
			\item{With an accessible user interface, the user will be willing to use and explore the application}
			\item{By having a friendly interface for the user, they can see everything clearly laid out, knowing where everything is}
		\end{itemize}

		\item{Store user templates locally on the mobile device}
		\begin{itemize}
			\item{By storing the user templates locally, we can avoid having to store large amounts of data on a server}
		\end{itemize}

		\item{Quick and efficient retrieval of data}
		\begin{itemize}
			\item{Quick response time won't leave the user waiting for data}
			\item{Efficient retrieval will make sure the server/client aren't expending too many resources}
		\end{itemize}
	\end{enumerate}

	\pagebreak

	\section*{User Scenarios}

	In the situation wherein a user is planning on flying to another country,
	the user would be able to select a travel plan for air travel.
	From here the user would be able to select a basic layout of what is required as a checklist,
	with the functionality to add and/or remove from it.
	They also would have the ability to create their own personal checklist/itinerary as suits their individual needs.
	From here the user would have the baseline checklist functionality with the ability to
	add, remove, check as ready, mark as unready, etc.
	in order to make sure they have everything packed and ready to go for their departure date.
	Along with this, there would be the accompanying push notifications to serve as reminders to
	let the user know leading up to and including their departure date something along the lines of
	``Be sure to remember your passport!'' in the event where they are crossing a border, and a passport is required.

	In the event where a user is attempting to plan a daily/weekly budget for their journey the application will contain a budgeting tab.
	During the users various travels, they will run into a scenario wherein they are
	entering a new country and must adapt to the countries currency.
	When the situation arises where the vendor takes both the countries legal tender,
	as well as the users currency they will be able to on the fly, do their own conversions.
	Following the user entering a shop, or approaching a stand in a country such as the Dominican Republic
	they will have the option of spending both the Dominican Peso, or the United States dollar.
	When the shop keeper asks them for 5 of the Dominican pesos as that is their preferred currency,
	the user will be able to access the application and enter in the Dominican peso value and directly compare said value to the United States dollar.
	When they then have the equivalent evaluation they may present their USD to the vendor in exchange for whatever it is they may be buying,
	and circumvent the awkwardness of the exchange.

	\pagebreak

\section*{UI Mockups}

	\begin{figure}[!hb]
		\centering
		\begin{subfigure}{0.15\linewidth}
			\includegraphics[width=\linewidth]{checklist-base.png}
			\caption{Packing checklist}
			\label{fig:checklist-base}
		\end{subfigure}
		\begin{subfigure}{0.15\linewidth}
			\includegraphics[width=\linewidth]{checklist-add.png}
			\caption{Adding to a checklist}
			\label{fig:checklist-add}
		\end{subfigure}
		\begin{subfigure}{0.15\linewidth}
			\includegraphics[width=\linewidth]{checklist-remove.png}
			\caption{Removing from a checklist}
			\label{fig:checklist-remove}
		\end{subfigure}
		\caption{Creating a packing list behaves much like a digital todo list.}
		\label{fig:checklist}
	\end{figure}

	\begin{figure}
		\centering
		\includegraphics[scale=0.15]{budget-base.png}
		\caption{The budgeting tab}
		\label{fig:budget}
	\end{figure}

    The pictures in fig.~\ref{fig:checklist} show the checklist feature.
	Checking an item changes the colour to easily see unchecked vs checked items.
    Items have a number, which represents an amount.
    The bottom tab has out app features laid out for easy navigation.
    Swipe left on items to remove them from the list.
    Adding an item lets you add an item name and amount.
    Pressing and holding down on an item lets you edit it.

    The budget section (fig.~\ref{fig:budget}) gives you easy access to a currency conversion calculator.
    You can enter a budget for your trip.
    When you add expenses it takes away from your total.
\end{document}
