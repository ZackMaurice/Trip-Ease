\documentclass[12pt]{article}

\usepackage{graphicx}
\usepackage{subcaption}
\usepackage{hyperref}

\title{TripEase}
\author{
Benjamin Karstad
\texttt{- 101092172}
\and
Conor MacQuarrie
\texttt{- 100824788}
\and
Zacharaia Maurice
\texttt{- 101116412}
\and
Graham Shannon
\texttt{- 100970902}
}

\begin{document}

	\maketitle

	\section*{Dev Logs Hyperlink}
		\href{https://github.com/theArcticGiant/comp3004-f20/tree/master/app/src/docs/dev logs}{Dev Logs}


	\section*{Functional Properties}

	\begin{itemize}

		\item{Packing Checklist}
		For our checklist we implemented everything we had originally set out to do.
		The checklist allows users to add items to the list. Users can remove items by swiping left on them.
		Users can keep tracks of which items they have packed with a checkbox associated with each item. Users are given a set of common items for different types of trips.
		They can load from these templates, add their own custom templates and/or delete any templates they don't wish to keep.

		\item{Itinerary/Planner}
			\begin{itemize}
				\item Allows for scheduling and planning of travel events
				\item Users can put notes on their planned activity, e.g. specific things to do on the trip
				\item Notifications will arrive on your device for any task on the same day as the task itself
				\item Notifications are comprised of the app name (Trip-Ease), name of the task, and the notes the user has written about the task
				\item Users can remove any planned task that has already passed, or that they no longer wish to have in their plans
			\end{itemize}
		\item{Budgeting}
			Fully implemented, user can create budget items that will count against their total budget.
			Items can added and deleted and the changes are reflected locally and on Firebase.

		\item{Misc Utilities}

		\begin{itemize}
			\item Weather
			\item Local emergency numbers
			\item Translation \& Common Phrases
			\\
			\\
			\\
			We decided that these were no longer required in Trip-Ease, since for most mobile phones there is already a weather app, there is an emergency button, as well as automatic dialling for emergency numbers.\\
			As for translation and common phrases, we felt that this was not needed and instead opted to change the Miscellaneous fragment into an account customization screen.
		\end{itemize}

		\item{Offline Saving}

	\end{itemize}

	\section*{Non-Functional Properties}


	\begin{enumerate}
		\item{User-friendly and accessible user interface}
		Our user interface is a very simple layout that users will be immediately familiar with it. Navigation between features is done through a bottom navigation bar.
		Obvious functionality (adding items/templates) is done through labelled buttons. Removing items is done through swiping left.
		A user jumping into our application will have very if (if any) questions about how to use our application.
		We feel that we accomplished implementing all the functionality that we wanted, while maintaining a simple and familiar user interface.

		\item{Store user templates locally on the mobile device}
		\begin{itemize}
			\item Templates are all stored locally on the user's mobile device, so the user can select them no matter if they have a wireless internet connection or not
			\item After a user selects a template, it adds the items into the user's checklist
		\end{itemize}

		\item{Quick and efficient retrieval of data}
			Firebase integration is remarkably responsive, changes are saved and loaded quickly.
	\end{enumerate}

	\pagebreak

\end{document}
